\documentclass{article}
\usepackage{amsmath,amssymb}
\usepackage{fullpage}
\usepackage{enumerate}
\usepackage{tikz}

\newcommand{\dsum}{\displaystyle\sum}
\newcommand{\dint}{\displaystyle\int}
\newcommand{\abs}[1]{\displaystyle\left\lvert#1\right\rvert}
\newcommand{\dbcup}{\displaystyle\bigcup}
\newcommand{\dbcap}{\displaystyle\bigcap}
\newcommand{\dcup}{\displaystyle\cup}
\newcommand{\dcap}{\displaystyle\cap}
\newcommand{\Pb}{\mathbb{P}}
\newcommand{\Rb}{\mathbb{R}}
\newcommand{\Eb}{\mathbb{E}}
\newcommand{\soln}[1]{\textbf{Solution}:#1}
\newcommand{\bkt}[1]{\left(#1\right)}
\title{MA2040: Probability, Statistics and Stochastic Processes\\
Problem Set-V}
\author{Sivaram Ambikasaran}
\begin{document}
	\maketitle
	\begin{enumerate}
		\item
		Let $\{X_i\}_{i=1}^{64}$ be a random sample from a normal distribution with mean $\mu = 50$ and variance $\sigma^2=16$. Find $\Pb\bkt{49 < X_8 < 51}$ and $\Pb\bkt{49 < \overline{X} < 51}$.\\
		\soln{
		\begin{align}
			\Pb\bkt{49 < X_8 < 51} & = \Pb\bkt{\dfrac{49-50}4 < \dfrac{X_8-50}4 < \dfrac{51-50}4} = \Phi\bkt{1/4} - \Phi(-1/4)\\
			& = 2\Phi\bkt{1/4}-1 = 2 \times 0.59871-1 = 0.19742
		\end{align}
		\begin{align}
			\Pb\bkt{49 < \overline{X} < 51} & = \Pb\bkt{\dfrac{49-50}{4/\sqrt{64}} < \dfrac{\overline{X}-50}{4/\sqrt{64}} < \dfrac{51-50}{4/\sqrt{64}}} = 2 \Phi\bkt{2}-1 = 0.9545
		\end{align}
		}
		\item
		Let $Y = X_1+X_2+\cdots+X_{15}$ be the sum of a random sample of size $15$ from the population whose probability density function is given by
		$$f_X(x) = \begin{cases}
		kx^2 & \abs{x} \leq 1\\
		0 & \text{otherwise}
		\end{cases}$$
		What is the approximate value of $\Pb\bkt{-0.3<Y<1.5}$ when one uses Central Limit Theorem?\\
		\soln{
		We have $k=3/2$. Mean is $0$. Variance is $0.6$. From CLT, we have
		\begin{align}
			\Pb\bkt{-0.3<Y<1.5} & = \Pb\bkt{-0.3/\sqrt{0.6 \times 15} < Z < 1.5/\sqrt{0.6 \times 15}} = \Phi\bkt{0.5} - \Phi\bkt{-0.1}\\
			& = 0.69146-0.46017\\
			& = 0.23129
		\end{align}}
		\item
		Light bulbs are installed successively into a socket. If we assume that each bulb has a mean life of $2$ months with a standard deviation of $0.25$ months, what is the probability that $40$ bulbs will last for at least $7$ years?\\
		\soln{
		Let $X_i$ be the life of the $i^{th}$ bulb in months. We want $S_{40} = \dsum_{i=1}^{40}X_i \geq 84$. Hence,
		$$\Pb\bkt{S_{40} \geq 84} = \Pb\bkt{\dfrac{S_{40}-40 \times 2}{0.25 \times \sqrt{40}} \geq \dfrac{84-40 \times 2}{0.25 \times \sqrt{40}}} = 1- \Phi\bkt{2.529} = 1-0.99413 = 0.00587$$
		}
		\item
		A random sample of size $36$ is taken from the population whose pdf is given by $f_X(x) = 2e^{-2x}$, for $x \geq 0$. If $\overline{X}$ denotes the sample mean, find $\Eb\bkt{\overline{X}}$ and $\Eb\bkt{\overline{X}^2}$. Also find the probability that $\overline{X} \in \left[\frac14,\frac34\right]$.\\
		\soln{
		We have the mean of the distribution to be $1/2$ and variance to be $1/4$. Hence, $\Eb\bkt{\overline{X}} = 1/2$ and $\Eb\bkt{\overline{X}^2} = 1/4/36 + 1/4 = \dfrac{37}{144}$. The desired probability is again computed by normal approximation, which gives us $0.9974$.
		}
		\item
		In order to estimate $f$, the true fraction of smokers in a large population, Alvin selects $n$ people at random. His estimator $M_n$ is obtained by dividing $S_n$ (the number of smokers in his sample) by $n$, i.e., $M_n=S_n/n$. Alvin chooses the sample size $n$ to be the smallest possible number for which the Chebyshev inequality yields a guarantee that 
		$$\Pb \bkt{\abs{M_n-f} \geq \epsilon} \leq \delta$$
		where $\epsilon$ and $\delta$ are some prespecified tolerances. Determine how the value of $n$ recommended by the Chebyshev inequality changes in the following cases.
		\begin{enumerate}
			\item
			The value of $\epsilon$ is reduced to half its original value.\\
			\soln{
			Goes up by a factor of $4$.
			}
			\item
			The value of $\delta$ is reduced to half its original value.\\
			\soln{
			Goes up by a factor of $2$.
			}
		\end{enumerate}
		\item
		Before starting to play the roulette in a casino, you want to look for biases that you can exploit. You therefore watch $100$ rounds that result in a number between $1$ and $36$, and count the number of rounds for which the result is odd. If the count exceeds $55$, you decide that the roulette is not fair. Assuming that the roulette is fair, find an approximation for the probability that you will make the wrong decision. (HINT: Central Limit Theorem)\\
		\soln{
		Let $S$ be the number of times that the result was odd. This is a binomial random variable, with $n=100$ and $p=0.5$. This gives us $\Eb\bkt{S} = 50$ and $\text{Var}\bkt{S} = 25$. From normal approximation, we obtain that
		$$\Pb\bkt{S > 55} = \Pb \bkt{\dfrac{S-50}5 > \dfrac{55-50}5} = 1- \Phi(1) = 0.1587$$
		}
		\item
		\textbf{Proof of Central Limit Theorem}: Let $\{X_i\}_{i=1}^n$ be a sequence of independent identically distributed random variables with mean zero, with variance, say $\sigma^2$, and associated transform $M_X(s)$. We assume that $M_X(s)$ is finite for $s \in \bkt{-d,d}$, for some $d \in \Rb^+$. Let
		$$Z_n = \dfrac{X_1+X_2+\cdots+X_n}{\sigma \sqrt{n}}$$
		\begin{enumerate}
			\item
			Show that the transform associated with $Z_n$ is given by
			$$M_{Z_n}(s) = \bkt{M_x\bkt{\dfrac{s}{\sigma\sqrt{n}}}}^n$$
			\soln{Immediate from definition of mgf.}
			\item
			Write the first two terms and the error term of the Taylor series expansion of $M_X$ around $s=0$.
			\item
			From the above conclude that
			$$\lim_{n \to \infty} M_{Z_n}(s) = e^{s^2/2}$$
			for all $s$. And note that $e^{s^2/2}$ is the moment generating function of a standard normal random variable.
		\end{enumerate}
	\end{enumerate}
\end{document}