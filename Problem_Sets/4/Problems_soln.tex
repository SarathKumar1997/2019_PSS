\documentclass{article}
\usepackage{amsmath,amssymb}
\usepackage{fullpage}
\usepackage{enumerate}
\usepackage{tikz}

\newcommand{\dsum}{\displaystyle\sum}
\newcommand{\dint}{\displaystyle\int}
\newcommand{\abs}[1]{\displaystyle\left\lvert#1\right\rvert}
\newcommand{\dbcup}{\displaystyle\bigcup}
\newcommand{\dbcap}{\displaystyle\bigcap}
\newcommand{\dcup}{\displaystyle\cup}
\newcommand{\dcap}{\displaystyle\cap}
\newcommand{\Pb}{\mathbb{P}}
\newcommand{\Eb}{\mathbb{E}}
\newcommand{\bkt}[1]{\left(#1\right)}
\newcommand{\soln}[1]{\\ \textbf{Solution}:#1\\}
\title{MA2040: Probability, Statistics and Stochastic Processes\\
Problem Set-IV}
\author{Sivaram Ambikasaran}
\begin{document}
	\maketitle
	\begin{enumerate}
		\item
		Let $X$ and $Y$ have joint pdf $f_{X,Y}(x,y) = \begin{cases}
		6 e^{-\bkt{2x+3y}} & x,y \geq 0\\
		0 & \text{otherwise}
		\end{cases}$
	\begin{itemize}
		\item
		Are $X$ and $Y$ independent?\\
		\soln{
		We have
		$$f_X(x) = \dint_0^{\infty}f_{X,Y}(x,y)dy = 2e^{-2x}$$
		$$f_Y(y) = \dint_0^{\infty}f_{X,Y}(x,y)dx = 3e^{-3y}$$
		We have $f_{X,Y} = f_X \cdot f_Y$. Hence, $X$ and $Y$ are independent.
		}
		\item
		Find $\mathbb{E}\bkt{Y \mid X>2}$\\
		\soln{
		Since $X$ and $Y$ are independent, conditioning on $X$ doesn't affect $Y$. Hence, $$\mathbb{E}\bkt{Y \mid X>2} = \mathbb{E}\bkt{Y} = \dint_0^{\infty}3ye^{-3y}dy = \dfrac13$$
		}
		\item
		Find $\Pb\bkt{X > Y}$\\
		\soln{
		$$\Pb\bkt{X > Y} = \dint_0^{\infty} \Pb\bkt{X>y}f_Y(y)dy = \dint_0^{\infty}\bkt{1-F_X\bkt{y}}f_Y(y)dy = 1-3\dint_0^{\infty}\bkt{1-e^{-2y}}e^{-3y}dy = 3/5$$
		}
	\end{itemize}
	\item
	Let $X \sim \text{Uniform}(1,2)$ and given $X=x$, $Y$ follows an exponential distribution with $Y \sim \text{EXP}\bkt{\lambda=x}$. Find the covariance of $X$ and $Y$.\\
	\soln{
	We have
	$$f_{XY}\bkt{x,y} = f_{Y \mid X}f_X = xe^{-xy}$$
	This gives us
	$$\text{Cov}\bkt{X,Y} = \Eb\bkt{XY}-\Eb\bkt{X}\Eb\bkt{Y} = 1-3/2 \ln(2)$$
	}
	\item
	Let $X$ and $Y$ be independent standard normal random variables. Find covariance of $Z$ and $W$, where $Z = 1+X+XY^2$ and $W = 1+X$.\\
	\soln{
	$\Eb\bkt{Z} = 1$ and $\Eb{W} = 1$. Further, $\Eb\bkt{ZW} = \Eb\bkt{1+2X+X^2 + XY^2 + X^2Y^2} = 3$.
	Hence, covariance is $2$.
	}
	\item
	A fair die is rolled $n$ times. Let $X$ be the number of $1$'s and $Y$ be the number of $2$'s. Find the correlation coefficient $\rho(X,Y)$.\\
	\soln{
	Let $Z=X+Y$. $X$, $Y$ are binomial random variables with $p=1/6$, while $Z$ is a binomial random variable with $p=1/3$. Hence, $\text{Var}(X) = \text{Var}(Y) = 5n/36$ and $\text{Var}(Z) = 2n/9$. Hence,
	$$\text{Cov}\bkt{X,Y} = \dfrac12\bkt{2n/9-5n/18} = -n/36$$
	Hence, $\rho = -1/5$.
	}
	\item
	Let $32$ people sit around a round table. Each person tosses a fair coin. Anyone whose outcome is different from both his neighbors is taken out from the group. If $X$ is the total number of such persons, find variance of $X$.\\
	\soln{
	Consider the random variable $P_i$, which takes a value $1$, when the $i^{th}$ person's outcome differs from his both neighbors and takes the value $0$, when the outcomes of him matches with both his neighbors. We see that $P_i = 1$ with a probability of $1/4$ and $P_i=0$ with a probability of $3/4$. Now we have $X = \dsum_{i=1}^{32} P_i$. $X$ is \textbf{not a binomial random variable}, since for instance, $P_i$ and $P_{i+1}$ are dependent, (in fact they are also correlated), since
	$$\text{Cov}\bkt{P_i P_{i+1}} = \Eb\bkt{P_i P_{i+1}} - \Eb\bkt{P_i} \Eb\bkt{P_{i+1}} = 1/8-1/4 \times 1/4 = 1/16$$
	Hence, we have that
	$$\text{Var}\bkt{X} = \text{Var}\bkt{\dsum_{i=1}^{32} P_i} = \dsum_{i=1}^{32}\text{Var}\bkt{P_i} + \dsum_{i \neq j} \text{Cov}\bkt{P_iP_j}$$
	Note that if $i$ and $j$ are apart by more than $2$, i.e., if $\abs{i-j}\pmod{32} \geq 3$, then $P_i$ and $P_j$ are independent. Hence, $\text{Cov}\bkt{P_iP_j} = 0$ for $\abs{i-j}\pmod{32} \geq 3$. The only non-zero covariances involving $P_i$ are along with $P_{i\pm1}$ and $P_{i \pm 2}$. (The indices $i\pm1$ and $i\pm2$ will be interpreted $\pmod{32}$ always). We have
	$$\text{Cov}\bkt{P_i P_{i \pm 1}} = \Eb\bkt{P_i P_{i \pm 1}} - \Eb\bkt{P_i}\Eb\bkt{P_{i \pm 1}} = 1/16$$
	$$\text{Cov}\bkt{P_i P_{i \pm 2}} = \Eb\bkt{P_i P_{i \pm 2}} - \Eb\bkt{P_i}\Eb\bkt{P_{i \pm 2}} = 1/2^4 - 1/4 \times 1/4 = 0$$
	Hence, we obtain
	$$\text{Var}\bkt{X} = \dsum_{i=1}^{32} \bkt{\text{Var}\bkt{P_i} + \text{Cov}\bkt{P_i P_{i+1}} + \text{Cov}\bkt{P_i P_{i-1}} + \text{Cov}\bkt{P_i P_{i+2}} + \text{Cov}\bkt{P_i P_{i-2}}}$$
	This gives us
	$$\text{Var}\bkt{X} = \dsum_{i=1}^{32} \bkt{\dfrac14 \times \dfrac34 + \dfrac1{16} + \dfrac1{16}} = 10$$
	}
	\item
	The moment generating function of a random variable $X$ is given by
	$$M_X(s) = \dfrac2{2-s}, \,\,\, \forall s \in (-\infty,2)$$
	Find the distribution of $X$.\\
	\soln{
	$X$ is exponential with $\lambda=2$.}
	\item
	Let $X \sim \text{Binomial}(n,p)$ and $Y \sim \text{Binomial}(m,p)$ be independent random variables. Show that $X+Y \sim \text{Binomial}(m+n,p)$.\\
	\soln{Use the MGF to conclude.}
	\item
	Show that the function $e^{-s}$ is a moment generating function.\\
	\soln{
	Consider the random variable which takes only the value $-1$, i.e., $\Pb\bkt{X=-1}=1$. The MGF is $e^{-s}$.
	}
	\item
	Show that if $M(s)$ is a moment generating function, so is $M(cs)$.\\
	Soln{
	If $M(s)$ is the mgf of $X$, then consider the random variable variable $Z = X/c$.
	}
	\item
	Show that if $M(s)$ is a moment generating function, $cM(s)$ cannot be a moment generating function for $c \neq 1$.\\
	\soln{
	The mgf needs to be $1$ at $s=0$.
	}
	\item
	Show that if $M(s)$ is a moment generating function, then $e^{-s}M(s)$ is also a moment generating function.\\
	\soln{
	If $M(s)$ is the mgf of $X$, then consider the random variable variable $Z = X-1$.
	}
	\item
	The moment generating function of a random variable $X$ is
	$$M_X(s) = \dfrac{e^{-2s}}6 + \dfrac{e^{-s}}3 + \dfrac{e^{s}}4 + \dfrac{e^{2s}}4$$
	Show that $P\bkt{\abs{X} \leq 1} = \dfrac7{12}$.\\
	\soln{
	This is a discrete random variable.
	$$\Pb\bkt{X=k}=\begin{cases}
	1/6 & \text{ if }X=-2\\
	1/3 & \text{ if }X=-1\\
	1/4 & \text{ if }X=1\\
	1/4 & \text{ if }X=2
	\end{cases}$$
	}
	\end{enumerate}
\end{document}